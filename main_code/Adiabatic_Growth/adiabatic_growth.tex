\documentclass[10pt]{article}
\usepackage{amsmath}
\usepackage{graphicx}

\title{Adiabatic Growth of Black Holes in a Plummer Sphere}
\author{Joseph D. MacMillan}
\date{May 2023}

\begin{document}

\maketitle

\section{Spherical Collisionless Systems}

In a collisionless system, the constituent particles (e.g., stars or dark matter particles) move under the influence of the smooth gravitational potential of the system $\Phi(\textbf{r}, t)$ (created by all other particles); that is, particles don't react to other particles, but rather the entire system itself.  In this case, the state of the system can be described by a distribution function (DF) $f(\textbf{r}, \textbf{v}, t)$; the quantity $f(\textbf{r}, \textbf{v}, t)\, d^3\textbf{r} d^3 \textbf{v}$ is the number (or mass) of particles having positions in the volume $d^3\textbf{r}$ centred on $\textbf{r}$ and having the range of velocities $d^3 \textbf{v}$ centred on $\textbf{v}$.

Any physical quantity can be calculated from the DF; for example, the density is
\[
\rho(\textbf{r}) = \int f d^3\textbf{v},
\]
and various moments of the velocity components $v_i$ can be computed from
\[
\langle v_i^m v_j^n \rangle = \frac{1}{\rho} \int v_i^m v_j^n f \, d^3\textbf{v},
\]
Given the density, the gravitational potential can be found from Poisson's equation,
\[
\nabla^2 \Phi = 4\pi G \rho(\textbf{r}),
\]
and once we have the potential, the motion of particles is given by Newton's second law.

Now, in the case that the DF is \emph{steady-state} (no time dependence) and spherically symmetric in all its properties, the system will admit four integrals of motion:  the energy (per unit mass) $E = v^2 /2 + \Phi$, and the three components of the angular momentum $\textbf{J}$.  Furthermore, the DF will depend only on the energy and the magnitude of the angular momentum:  $f = f(E, J)$ only.  

Let's write the velocity in terms of a \emph{radial} component and a \emph{tangential} (or \emph{transverse}) component: 
\[
v^2 = v_r^2 + v_t^2.
\]
The transverse component is related to the angular momentum, since from
\[
\textbf{J} = \textbf{r} \times \textbf{v}
\]
we have
\begin{equation}
J = rv_t,
\end{equation}
so the energy can be written 
\[
E = \frac{v_r^2}{2} + \frac{J^2}{2r^2} + \Phi,
\]
or, rearranged for the radial velocity, we have
\begin{equation}
v_r = \sqrt{2 (E - \Phi) - J^2/r^2}.
\end{equation}
Note that since the radial velocity can't be imaginary, the angular momentum has a maximum possible value 
\begin{equation}
J_\text{max} = \sqrt{2r^2 (E - \Phi)}.
\end{equation}

Finally, we're ready to transform from $(\textbf{r}, \textbf{v})$ to $(E, J)$ space.  In terms of the energy and angular momentum, the density becomes
\begin{equation}
\rho(r) = 4 \pi \int_{\Phi(r)}^{\Phi(\infty)} dE \int_0^{J_\text{max}} \frac{J\, dJ}{r^2 v_r} \, f(E, J),
\end{equation}
and the velocity moments are
\begin{equation}
\langle v_r^m v_t^n \rangle = \frac{4 \pi}{\rho} \int_{\Phi(r)}^{\Phi(\infty)} dE \int_0^{J_\text{max}} \frac{J\, dJ}{r^2 v_r} v_r^m v_t^n f(E, J).
\end{equation}
Poisson's equation, for spherical symmetry, becomes
\begin{equation}
\Phi(r) = 4 \pi G \int_0^r \rho(s) s \, ds - \frac{GM(r)}{r},
\end{equation}
where $M(r)$ is the total mass within a radius $r$,
\begin{equation}
M(r) = 4 \pi \int_0^r \rho(s) s^2 \, ds.
\end{equation}

\section{The Plummer Sphere}

We're investigating the growth of supermassive black holes in dark matter halos, and to start we need a simple model of the halo.  Although it's not a \emph{great} model, the Plummer sphere will do for now; it's defined by the DF
\begin{equation}
f(E, J) = \frac{24 \sqrt{2}}{7 \pi^3} \frac{a^2}{G^5 M^4} (-E)^{7/2},
\end{equation}
where $a$ is a scale radius and $M$ is the total mass of the system.  Normally we'll take $G = M = a$.  The gravitational potential of the Plummer sphere is
\begin{equation}
\Phi(r) = - \frac{GM}{a} \frac{1}{\sqrt{1 + r^2/a^2}}
\end{equation}
and the density is
\begin{equation}
\rho(r) = \frac{3}{4\pi} \frac{M}{a^3} \frac{1}{(1 + r^2/a^2)^{5/2}}.
\end{equation}
From the above equations, we can calculate 
\begin{equation}
\langle v_r^2 \rangle = \frac{GM}{6a} \frac{1}{\sqrt{1 + r^2/a^2}}
\end{equation}
and 
\begin{equation}
\langle v_t^2 \rangle = \frac{GM}{3a} \frac{1}{\sqrt{1 + r^2/a^2}}.
\end{equation}



\section{Adiabatic Growth of Black Holes}

See thesis for details, too lazy to write it out here.  The Jupyter notebook also has details.  Just the plot:

\includegraphics[scale=0.55]{code/fig_results.pdf}

\end{document}
